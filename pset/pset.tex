\documentclass[12pt]{amsart}

%Below are some necessary packages for your course.
\usepackage{amsfonts,latexsym,amsthm,amssymb,amsmath,amscd,euscript}
\usepackage{framed}
\usepackage{fullpage}
\usepackage{hyperref}
    \hypersetup{colorlinks=true,citecolor=blue,urlcolor =black,linkbordercolor={1 0 0}}

\newenvironment{statement}[1]{\smallskip\noindent\color[rgb]{1.00,0.00,0.50} {\bf #1.}}{}
\allowdisplaybreaks[1]

%Below are the theorem, definition, example, lemma, etc. body types.

\newtheorem{theorem}{Theorem}
\newtheorem*{proposition}{Proposition}
\newtheorem{lemma}[theorem]{Lemma}
\newtheorem{corollary}[theorem]{Corollary}
\newtheorem{conjecture}[theorem]{Conjecture}
\newtheorem{postulate}[theorem]{Postulate}
\theoremstyle{definition}
\newtheorem{defn}[theorem]{Definition}
\newtheorem{example}[theorem]{Example}

\theoremstyle{remark}
\newtheorem*{remark}{Remark}
\newtheorem*{notation}{Notation}
\newtheorem*{note}{Note}

% You can define new commands to make your life easier.
\newcommand{\BR}{\mathbb R}
\newcommand{\BC}{\mathbb C}
\newcommand{\BF}{\mathbb F}
\newcommand{\BQ}{\mathbb Q}
\newcommand{\BZ}{\mathbb Z}
\newcommand{\BN}{\mathbb N}

% We can even define a new command for \newcommand!
\newcommand{\nc}{\newcommand}

% If you want a new function, use operatorname to define that function (don't use \text)
\nc{\on}{\operatorname}
\nc{\Spec}{\on{Spec}}

\title{Problem Set $n$} % IMPORTANT: Change the problemset number as needed.
\date{\today}

\begin{document}

\begin{framed}
\maketitle 

\vspace*{-0.25in} \noindent
PHP0000 \hfill Daniel C. Posmik \\ 
Prof. John Doe \hfill{Pronouns: He/Him/His} \\
\today \hfill {\href{mailto:daniel_posmik@brown.edu}{{\tt daniel\_posmik@brown.edu}}}
\end{framed}

\begin{statement}{1}
    This is the problem statement. To help your graders, always preface your solution with the problem statement. Also, putting the problem statement in a different color helps your graders distinguish between problem and solution. We use the {\tt statement} environment to help us do this.
\end{statement}

\begin{proof}
    Type your solution in this body. Feel free to use definitions, lemmas, and examples as needed in your proofs; e.g.:
    \begin{defn}
        Define $\exp(x)$ for $x \in \BR$ to be the value of $$\sum_{i = 0}^\infty\frac{x^i}{i!}.$$
    \end{defn}
    As in the above definition, use separate equations rather than in-line equations as much as possible. In general, if your mathematical expression takes up more than an inch on paper, you should probably put it in its own line. This makes your problemset more readable. Use equation arrays for lists of equalities:
    \begin{align*}
        0 &= 0 + 0 + 0 + 0 + \dots\\
        &= (1 - 1) + (1 - 1) + \dots \\
        &= 1 + (-1 + 1) + (-1 + 1) + \dots \\
        &= 1 + 0 + 0 + 0 \dots \\
        &= 1.
    \end{align*}

    If you need to list things, use {\tt enumerate} or {\tt itemize}; e.g. Daily Schedule:
    \begin{enumerate}
        \item Do the problem set.
        \item Do the problem set.
        \item Do the problem set.
    \end{enumerate}
    And {\tt itemize} gives you bullet points.
\end{proof}

\begin{statement}{2}
    Show that there are no nontrivial integer solutions to $a^n + b^n = c^n$ when $n\ge 3$ is an integer.
\end{statement}

\end{document}
